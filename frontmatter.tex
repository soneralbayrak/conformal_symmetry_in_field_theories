\chapter{Preface}
\sidebar{\vspace*{\baselineskip}}
\section{Remarks to the reader}
I would like to make a few points crystal clear:
\begin{enumerate}
	\item I do not update these notes on a regular basis.
	\item The contents are correct to the best of my knowledge, but I have not put the extra effort to make sure that everything is book-level correct; nevertheless, I try to put as many references as possible when relevant, so please make the most of it!
	\item I prefer a casual tone in my notes: if you don't like it, just deal with it! I believe this makes it easier for students to connect with written content, but it can also be a symptom for a midlife crisis of a theoretical physicist (I should be young enough for this to be correct though). 
	\item I believe that the level of this book is appropriate for an average senior undergraduate student, but I would dare say that it should be quite useful even for graduate students of theoretical physics. 
	\item Lastly, I provide several links as references here and there: I agree that this is a bad practice in academia and one should instead convert them to proper references in the bibliography. Nevertheless, it is faster for me to write this way and faster for the reader to just click on them where they appear, so I'll keep this practice. My upfront apologies if the links get broken in time: hopefully a snapshot will have been available here \href{https://web.archive.org/}{https://\\web.archive.org/} (it would be rather amusing if this site itself becomes unavailable).
\end{enumerate}
\section{About the format of the book}
This book is publicly available online, and anyone can simply download and read it on their computer. Nevertheless, it is a lot easier on eyes to read a book on paper (or epaper), so I suspect many readers will simply print this book. To make this book convenient for all types of readers, I need to make a few design choices\footnote{For instance, I provide all links in their explicit form (not like \href{https://www.google.com/}{google} but as \href{https://www.google.com/}{https://www.google.com/}) so that the book on an ereader without a browser or its printed version is as useful as its digital version.} and the most important of it is the format of this book.

The most convenient paper that most students have easy access to is A4 (or letterpaper which is very similar in size): so if they choose to print this book, their easiest and cheapest option would be to use A4 paper. However that paper has been historically designed for typewriters which use large monospaced fonts, hence is not really appropriate for a digitally prepared book. Indeed, if you check your favorite-to-read book, you will most likely see that it uses a smaller paper size, with proportionally spaced small fonts.

\beginfullwidth
\hphantom{\indent} What is wrong with using a large paper? It is empirically known that a document is properly legible if there are around 60-75 characters on a line: if there are more characters, it becomes harder to read and the reader may end up re-reading same line over and over again (doubling). When used with typewriter, A4 paper indeed has appropriate number of characters one a line, but as stated previously, this is not the optimal setup with the digital fonts, hence making A4 paper \emph{too large for digital books}.
\\\\
{\ttfamily \hphantom{\indent} How to solve the problem that A4 is too large for a digital book?\newline Obviously, we can use a typewriter font for which A4 is historically\newline intended in the first place! However, such monospaced fonts (Courier\newline being another example) are not aesthetically pleasing and do not\newline belong to modern texts!}
\\\\
{\LARGE \hphantom{\indent} Of course, we can go with a modern proportionally spaced fonts but make the font size large enough such that a line has few enough characters for it to be easily legible. Although a better one than using monospaced fonts, this is still a suboptimal solution to the problem at hand...}
%
\begin{DoubleSpace}
	Another solution which is somehow popular around the institutions is to use double-spacing among the lines. Indeed, regulations for master and doctorate thesis of various universities include compulsory large spacing among the lines, such as one and a half spacing or double spacing: you can also see this in my master and doctorate thesis: \href{https://arxiv.org/abs/1602.07676}{https://arxiv.org/abs/1602.07676}, and \href{https://arxiv.org/abs/2107.13601}{https://arxiv.org/abs/2107.13601}. Although this method can indeed prevent doubling to a degree, it is neither an aesthetic nor an efficient solution.
\end{DoubleSpace}
\vspace*{-\baselineskip}
	\begin{multicols}{2}
	A somehow better solution than those listed above is to use multiple columns in the document. Indeed, this is the traditional approach in magazines and newspapers, and is immediately applicable in academic papers and manuscripts as well. However, A4 paper is not really big enough to have two columns of text with around 60-75 characters (let alone three or more columns), and although one can go with smaller font sizes to make it more legible digitally, the printed version would still be hard to read either way (proper number of small font characters, or few number of normal font characters).
\end{multicols}
\endfullwidth 
\begincenter
\qquad Common text editors such as Microsoft Word either go with one of the solutions above or do not solve the problem at all. On the contrary, \LaTeX\;templates default to choose another approach: they stick to a modern font with an appropriate spacing and a single column, but they also increase the margins such that a line has proper number of characters. This is an ideal solution if the resultant text will be read digitally, however it leads to a waste of paper when printed.
\endcenter
\vspace*{\baselineskip}
\sidebar{\vspace*{50\baselineskip}}
\hphantom{\indent} In this book, we will not follow any of these design choices. Instead, we will go with the rather unorthodox \emph{Tufte style},\footnote{See \href{https://www.ctan.org/tex-archive/macros/latex/contrib/tufte-latex/}{https://www.ctan.org/tex-archive\\/macros/latex/contrib/tufte-latex/}} an asymmetric allocation of the text in the paper. Indeed, the main text will be in the left of the paper, whereas we have another block of text on the right dedicated to the \emph{sidenotes},\footnote{In the traditional layout, one usually uses endnotes, margin notes, or footnotes; in this paper, most ``notes'' will be sidenotes with occasional footnotes.\bottomnotemark}\bottomnotetext{Such as this one.} margin figures, and margin tables. We choose a rather narrow font family (\emph{libertine}) and arrange the margins such that the main text is of 26 pica width and side text is of 14 pica width: for the 11pt and 9pt font sizes, this corresponds to roughly 66 and 44 characters of the libertine font for main and side text blocks respectively.\footnote{For a nice discussion of these points along with the tools to compute approximate expected number of characters per line, see \href{https://ftp.cc.uoc.gr/mirrors/CTAN/macros/latex/contrib/memoir/memman.pdf}{https://ftp.cc.uoc.gr/mirrors/CTAN/macros\\/latex/contrib/memoir/memman.pdf}} Thus we have an ideally-sized main text block and acceptably-sized side text block for a modern proportionally spaced font in an A4 paper, and we do this without unnecessarily wasting the paper.\footnote{I would like to acknowledge the following nice discussion with which I started to learn more about these \emph{typographical} issues: \href{https://tex.stackexchange.com/questions/71172/why-are-default-latex-margins-so-big}{https://tex.stackexchange.com/questions/71172\\/why-are-default-latex-margins-so-big}.}

\section{Literature recommendations}
One can find many books, reviews, lecture notes, and articles about \emph{conformal symmetry}, \emph{conformal bootstrap program}, \emph{quantum theory of fields}, \emph{renormalization group}, \emph{critical phenomena}, \emph{AdS/CFT correspondence}, \emph{cosmological bootstrap}, \emph{flat space holography}, \emph{string theory} and many other related concepts to conformal symmetry and its usage in theoretical physics. You can go ahead and search these keywords and see how large a literature there exists for each and every one of these topics.

So finding a source for conformal symmetry or its usage is no problem, it is the other way around: there are simply too many resources and one can feel overwhelmed and find it hard to navigate through those articles to learn the subject. So below, I'll list some resources that I would recommend both to \naive younglings who know nothing about these subjects and to those physics-hardened veterans who have heard of/worked on some of these subjects but haven't really got the chance to look at conformal symmetry directly. Please note that the list is \emph{faaaaar} from being complete and is by no means supposed to include good resources and to exclude bad ones: I simply listed what I know (and would recommend) and it is extremely likely that there are reviews out there that would be far more suited yet are unbeknownst to me.

\begin{enumerate}
	\item The traditional referencing source for conformal symmetry is the so-called \emph{yellow book}, i.e. \emph{Conformal Field Theory} by Francesco, Mathieu, and Sénéchal. It is a rather complete book, i.e. it covers all the basics, however it is published in 1997 and hence does not include recent developments captured by \emph{conformal bootstrap program}. In short, conformal symmetry is a lot easier to solve in two spacetime dimensions and up until the last two decades all the major progress has been in two dimensions, hence naturally most of the book's content is for two dimensional conformal models.
	\item The de facto referencing source for the conformal bootstrap program (which aims to constraint or solve conformal models in higher dimensions, i.e. $d\ge 3$) is the review article \emph{The Conformal Bootstrap: Theory, Numerical Techniques, and Applications} by Poland (who happens to be my PhD advisor\footnote{Thus I may be biased towards his work due to more exposure.}), Rychkov, and Vichi. The article really does not get into details (and not really pedagogical), so it is not the best source to \emph{learn} stuff. However, it is an excellent collection of sources, so one can use it to find other sources for individual topics.
	\item The de facto lecture notes to learn the basics of the conformal symmetry is \emph{TASI Lectures on the Conformal Bootstrap} by Simmons-Duffin (who happens to be my host during my 1-year visit to Caltech\footnote{Thus I may be biased towards his work due to more exposure.}). Around similar times, Qualls shared his lecture notes \emph{Lectures on Conformal Field Theory} as well (which is less focused on bootstrap but more on 2d CFTs).\footnote{There is also the lecture notes \emph{Applied Conformal Field Theory} by Ginsparg. Honestly, I did not fully read the notes as they are rather focused on $2d$ CFTs which are not really my personal focus; however, some may find them quite useful.} A more recent review was given by Osborn, i.e. \emph{Lectures on Conformal Field Theories in more than two dimensions}, however they are somehow more technical and less pedagogical.
	\item Slava Rychkov maintains an active blog where he writes about his papers or his talks among other useful information, see \hyperref{https://sites.google.com/site/slavarychkov}{}{}{https: //sites.google.com/site/slavarychkov}. He also wrote lecture notes titled \emph{EPFL Lectures on Conformal Field Theory in D>= 3 Dimensions}, it is a bit older than Simmons-Duffin's lecture notes, but some may prefer the style. He also talked about the \emph{philosophy} of the bootstrap approach as the $27^{\text{th}}$ Ockham Lecture, see \hyperref{https://www.merton.ox.ac.uk/event/27th-ockham-lecture-reductionism-vs-bootstrap-are-things-big-always-made-things-elementary}{}{}{https://www.merton.ox.ac.uk/event/27th-ockham-lecture\\-reductionism-vs-bootstrap-are-things-big-always-made-\\things-elementary}.
	
	\item If you would like a \emph{hardcore} resource, then there is the \emph{holy book}.\footnote{See \hyperref{http://doi.org/10.1007/BFb0009678}{}{}{http://doi.org/10.1007/BFb0009678}.} It is written by Dobrev, Mack, Petkova's, and Todorov in 1977 and is still extremely relevant today. The only problem is that it is rather mathematical (and technical), and you may find it hard --- honestly I did not fully read the book myself, I only have used it as a reference to check stuff here and there. 
	\item If you would like a \emph{layman review} of the conformal bootstrap, then I'd suggest the nature article of Poland and Simmons-Duffin, i.e. \hyperref{https://doi.org/10.1038/nphys3761}{}{}{https://doi.org/10.1038/nphys3761}.
	
	\item If you prefer \emph{watching} to \emph{reading}, then there are recordings of excellent lectures in the summer schools of the conformal bootstrap collaboration.\footnote{See \hyperref{http://bootstrapcollaboration.com/activities}{}{}{http://bootstrapcollaboration.com/activities}} You can also view all videos of the collaboration at their Youtube channel:\footnote{See \hyperref{https://www.youtube.com/channel/UCgWLG2q2275RuUJ5eNSCCFA/videos}{}{}{https://www.youtube.com/channel/ UCgWLG2q2275RuUJ5eNSCCFA/videos}.} I personally advice 2017 \& 2018 Bootstrap School videos as they were rather introductory and pedagogical!
	
	\item These lectures will be mostly about conformal symmetry in $d\ge 3$ spacetime dimensions, hence $d=2$ conformal symmetry (and related Virasoro symmetry) are not our focus. For those interested in $d=2$ conformal symmetry (and its utility in String theory), I already named a few sources, but there are two books I would like to refer for sentimental reasons:  \emph{String and Symmetries}, edited by Gülen Aktaş, Cihan Saçlıoğlu, and Meral Serdaroğlu; and \emph{Conformal Field Theory: New Non-perturbative Methods In String And Field Theory}, edited by Yavuz Nutku, Cihan Saçlıoğlu, and Teoman Turgut. The former book consists of the proceedings of the Feza Gürsey Memorial Conference 1 (1994); and the latter book consists of lecture notes of  \emph{1998 Summer Research Semester on Conformal Field Theories, M(atrix) Models and Dualities}; both events held at Bogaziçi Üniversitesi-Tübitak Feza Gürsey Institute.
	
	\item Lastly, I can recommend the first two chapters of my thesis as \emph{an earnest attempt at a very pedagogical introduction} to conformal symmetry and the conformal bootstrap program. I tried to give a historical account of the subject, its basics, and recent developments. Even if you don't like it, you may find the references useful as I tried my best to organize and include as many references as possible. You can find it via the publisher's website\footnote{See \hyperref{https://www.proquest.com/docview/2557212384/2C5A80A2ADB447E7PQ}{}{}{https://www.proquest.com/docview/\\2557212384/2C5A80A2ADB447E7PQ}.} or via arXiv.\footnote{See \hyperref{https://arxiv.org/abs/2107.13601}{}{}{https://arxiv.org/abs/2107.13601}.}
\end{enumerate}